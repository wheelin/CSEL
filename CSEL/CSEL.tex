\documentclass[a4paper,12pt]{article}
\usepackage[french]{babel} 
\usepackage[T1]{fontenc}
%\usepackage[ansinew]{inputenc}
\usepackage[utf8]{inputenc}
\usepackage[top=3cm, bottom=3cm, left=2.3cm,right=2cm]{geometry}
\usepackage{graphicx}
\usepackage{color}
\usepackage{listings}
%\usepackage{marvosym}
%\usepackage{yfonts}
\usepackage[normalem]{ulem}
\usepackage{verbatim}
\usepackage{listings}
\usepackage{float}
%\renewcommand{\thesection}{\arabic{section}}
\usepackage{array} % pour les tableaux
\usepackage{amsmath} % pour les équations
\usepackage{float}
\usepackage{hyperref}	% crée des liens dans le pdf
\hypersetup{					% colorise les liens du pdf
  colorlinks=true,
  urlcolor=black
	citecolor=black,
  linkcolor=black,
  urlcolor=blue
}
\usepackage{url}			% change la police des url (utilisation : \url{http://asdf.ch})
\definecolor{dkgreen}{rgb}{0,0.6,0}
\definecolor{gray}{rgb}{0.5,0.5,0.5}
\definecolor{mauve}{rgb}{0.58,0.01,0.82}
%[babel=true]
\usepackage{csquotes}
\lstset{ %
  language=C,                % the language of the code
  basicstyle=\footnotesize,           % the size of the fonts that are used for the code
  numbers=left,                   % where to put the line-numbers
  numberstyle=\tiny\color{gray},  % the style that is used for the line-numbers
  stepnumber=1,                   % the step between two line-numbers. If it's 1, each line 
                                  % will be numbered
  numbersep=5pt,                  % how far the line-numbers are from the code
  backgroundcolor=\color{white},      % choose the background color. You must add \usepackage{color}
  showspaces=false,               % show spaces adding particular underscores
  showstringspaces=false,         % underline spaces within strings
  showtabs=false,                 % show tabs within strings adding particular underscores
  frame=single,                   % adds a frame around the code
  rulecolor=\color{black},        % if not set, the frame-color may be changed on line-breaks within not-black text (e.g. commens (green here))
  tabsize=2,                      % sets default tabsize to 2 spaces
  captionpos=b,                   % sets the caption-position to bottom
  breaklines=true,                % sets automatic line breaking
  breakatwhitespace=false,        % sets if automatic breaks should only happen at whitespace
  title=\lstname,                   % show the filename of files included with \lstinputlisting;
                                  % also try caption instead of title
  keywordstyle=\color{blue},          % keyword style
  commentstyle=\color{dkgreen},       % comment style
  stringstyle=\color{mauve},         % string literal style
  escapeinside={\%*}{*)},            % if you want to add a comment within your code
  morekeywords={*,...}               % if you want to add more keywords to the set
}


%en-tête
\usepackage{fancyhdr}
\lhead{PGa}
\chead{}
\rhead{\today}
\pagestyle{fancy}

% Title Page
\title{\Huge{\textsc{Construction de systèmes embarqués sous Linux}} \\ 
\Huge{\textbf{Rapport de laboratoire}} \\
\huge{Master HES-SO}}
\author{Émilie \textsc{Gsponer}, Grégory \textsc{Emery} }
\date{\today \\
version 1.0}

%-------------------------début du document-------------------------------------
\begin{document}

\maketitle % page de garde
\newpage
\tableofcontents % table des matières

\section{Introduction}
Ce rapport présente les résultats obtenus tout au long des travaux pratiques fournis durant le cours de CSEL1, construction de systèmes embarqués sous Linux. Le document est structuré en sections, représentant les séries d'exercices données, en sous-sections présentant les thèmes proposés pour les travaux et en sous-sous-sections pour les réponses à chacune des questions posées dans le document. \\
Ce cours est effectué avec la cible Odroid XU3\footnote{Lien : \url{http://www.hardkernel.com/main/products/prdt_info.php?g_code=G140448267127}} et U-Boot\footnote{Lien : \url{http://www.denx.de/wiki/U-Boot}} dans le cadre du cours de Master HES-SO en systèmes embarqués, orientation TIN et TIC.
\section{Série 1, exercice 5 : gestion de la mémoire, bibliothèques et fonctions utiles}
\textbf{Donnée : }Indiquer les différents alocateurs SLAB disponibles dans le noyau Linux pour la cible ORDOID-XU3
\begin{enumerate}
	\item SLAB : "as cache frendly as possible, benchmark frendly"
	\item SLOB : "as compact as possible"
	\item SLUB : "Simple and instruction cost counts. Superior Debugging. Defragmentation. Execution time friendly"
\end{enumerate}
Source : \url{https://www.google.ch/url?sa=t&rct=j&q=&esrc=s&source=web&cd=3&cad=rja&uact=8&ved=0CDEQFjACahUKEwiqj6GfhKbIAhWLXBoKHXDUAow&url=http%3A%2F%2Fwww.cs.berkeley.edu%2F~kubitron%2Fcourses%2Fcs194-24-S14%2Fhand-outs%2Fbonwick_slab.pdf&usg=AFQjCNENx6yGi8dVRbdR2Si1OpXE_NuNkg&sig2=ZdJ_jUWHIfO1qFIIikEyHA}

\end{document}


