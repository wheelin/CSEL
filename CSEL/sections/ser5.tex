\section{Optimisation : Performances}
\subsection{Installation de perf}
En suivant pas à pas les instructions de la donnée, perf a pu être installé sans aucun problème sur la machine virtuelle.
\subsection{Prise en main de perf}
\subsubsection{Exercice 1}
\textbf{Donnée : } Décrivez	brièvement	ce	que	sont	les	évènements	suivants	:
\begin{enumerate}
	\item instructions:
	\item cache-misses:
	\item branch-misses:
	\item L1-dcache-load-misses:
	\item cpu-migrations:
	\item context-switches:
\end{enumerate}

\subsubsection{Exercice 2}
\textbf{Donnée : } Compilez	(en	utilisant	le	Makefile	fourni)	et	exécutez	le	programme	situé	dans	le	dossier	exercices/01	
en	utilisant	la	commande	“perf	stat”
\\\\

\textbf{Exécution du code : } \\
Le code s'exécute jusqu'à ce qu'on l'interrompe avec ctrl-c par exemple. Un fois l'application interrompues, les mesures de performance s'affichent. Les mesures relevées sont les compteurs de context-switches et d'instruction ainsi que le temps d'exécution de l'application.\\
\begin{lstlisting}
lmi@csel1:~$ cd workspace/csel1/environment/exercices/01/
lmi@csel1:~/workspace/csel1/environment/exercices/01$ make
\end{lstlisting}

\begin{lstlisting}
# perf stat ./ex1 

Performance counter stats for './ex1':
...        
56 context-switches          #    0.001 K/sec                  
... 
10453003007 instructions              #    0.11  insns per cycle        
...
45.705296089 seconds time elapsed


\end{lstlisting}
On constate que l'exécution du code prend énormément de temps

\subsubsection{Exercice 3}
\textbf{Donnée : } Ouvrez	maintenant	le	fichier	“main.c”	et	analysez	le	code.
Ce	programme	contient	une	erreur	triviale	qui	empêche	une	utilisation	optimale	du	cache.	De	quelle	
erreur	s'agit-il	?\\\\
L'erreur vient du fait que l'on navigue dans le tableau de ligne en ligne. Cela donne énormément de travaille à la cache qui doit aller chercher dans la RAM à chaque incrément de colonne.
En navigant dans le tableau de colonne en colonne, la cache à moins de travaille, car elle peut charger une ligne entière, l'adresse ne change pas. Le temps d'exécution du code devrait être nettement plus rapide.\\

\textbf{Modification du code : } \\
\begin{lstlisting}
for (k = 0; k < 50; k++)
{
	for (i = 0; i < SIZE; i++)
	{
		for (j = 0; j < SIZE; j++)
		{
			//array[j][i]++;
			array[i][j]++;
		}
	}
}
\end{lstlisting}

\subsubsection{Exercice 4}
\textbf{Donnée : } Corrigez	“l'erreur”,	recompilez	et	mesurez	à	nouveau	le	temps	d'exécution	(soit	avec	perf	stat,	soit	
avec	la	commande	time).	Quelle	amélioration	constatez-vous	?\\\\
On constate que l'exécution de l'application est environ 15 fois plus rapide. La cache à moins besoin d'aller chercher des informations dans la RAM.\\\\
\textbf{Exécution du code : } \\
\begin{lstlisting}
# perf stat ./ex1 

Performance counter stats for './ex1':
...      
3 context-switches          #    0.001 K/sec                  
...  
6552198916 instructions              #    1.06  insns per cycle        
...
3.105906000 seconds time elapsed

\end{lstlisting}

\subsubsection{Exercice 5}
\textbf{Donnée : } Relevez	les	valeurs	du	compteur	L1-dcache-load-misses	pour	les	deux	versions	de	l'application.	Quelle	
facteur	constatez-vous	entre	les	deux	valeurs	?\\\\

\textbf{Exécution du code de base: } \\
\begin{lstlisting}
# perf stat -e L1-dcache-load-misses ./ex1 

Performance counter stats for './ex1':

1252652488 L1-dcache-load-misses                                       

45.638627964 seconds time elapsed
\end{lstlisting}

\textbf{Exécution du code modifié: } \\
\begin{lstlisting}
# perf stat -e L1-dcache-load-misses ./ex1 

Performance counter stats for './ex1':

78861576 L1-dcache-load-misses                                       

3.097171626 seconds time elapsed

\end{lstlisting}

On obtient un facteur d'environ 15.9 entre les deux exécution. Cela prouve que notre modification a une réelle influence sur la cache.

\subsubsection{Exercice 6}
\textbf{Donnée : } Lors	de	la	présentation	de	l'outil	perf,	on	a	vu	que	celui-ci	permettait	de	profiler	une	application	avec	
très	peu	d'impacts	sur	les	performances.	En	utilisant	la	commande	time,	mesurez	le	temps	d'exécution	
de	notre	application	ex1	avec	et	sans	la	commande	perf	stat.\\\\

\textbf{Exécution du code : } \\
\begin{lstlisting}

\end{lstlisting}

\subsection{Analyse et optimisation d'un programme}
Sur	la	base	du	programme	situé	dans	le	dossier	exercices/02/
\subsubsection{Exercice 1}
\textbf{Donnée : } Décrire	en	quelques	mots	ce	que	fait	ce	programme\\\\

\subsubsection{Exercice 2}
\textbf{Donnée : } Compilez	le	programme	à	l'aide	du	Makefile	joint.
Mesurez	le	temps	d'exécution\\\\

\textbf{Exécution du code : } \\
\begin{lstlisting}

\end{lstlisting}

\subsubsection{Exercice 3}
\textbf{Donnée : } Avant	la	fonction	int	main(),	ajoutez	la	méthode	suivante :
static int compare (const void* a, const void* b)
{return (short*)a - (short*)b;}
Avant	« long	long	sum	=	0; »,	ajoutez	le	code	suivant :
qsort(data, SIZE, sizeof(data[0]), compare);
Compilez	et	mesurez	le	temps	d'exécution	de	la	version	modifiée\\\\
\textbf{Emplacement du code : } \textit{}\\

\textbf{Exécution du code : } \\
\begin{lstlisting}

\end{lstlisting}

\subsubsection{Exercice 4}
\textbf{Donnée : } Vous	observez	sans	doute	une	nette	amélioration	sur	le	temps	d'exécution.
A	l'aide	de	l'outil	perf	et	de	sa	sous-commande	'stat',	en	utilisant	différents	compteurs	déterminez	
pourquoi	le	programme	modifié	s'exécute	plus	rapidement.\\\\
\textbf{Emplacement du code : } \textit{}\\

\textbf{Exécution du code : } \\
\begin{lstlisting}

\end{lstlisting}

\subsection{Parsing de logs apache}
\subsubsection{Exercice 1}
\textbf{Donnée : } Compilez	l'application	en	utilisant	le	Makefile	fourni	et	mesurez sur	la	machine	virtuelle son	temps	
d'exécution	avec	la	commande	“time”.\\\\

\textbf{Exécution du code : } \\
\begin{lstlisting}

\end{lstlisting}

\subsubsection{Exercice 2}
\textbf{Donnée : } Nous	allons	maintenant	profiler	l'application	avec	perf	record	avec	l'option	-g.
\begin{lstlisting}
$ perf record -g ./read-apache-logs access_log_NASA_Jul95_samples
\end{lstlisting}
L'exécution	de	cette	commande	doit	produire	un	fichier	de	résultat	perf,	nommé	perf.data.	Si	l'on	
exécute	une	nouvelle	fois	la	commande,	ce	fichier	sera	copié	vers	perf.data.old et	un	nouveau	
perf.data	correspondant	à	la	dernière	exécution	sera	créé.
A	quoi	sert	l'option	-g	?	En	quoi	son	résultat	être	utile	?\\\\

\textbf{Exécution du code : } \\
\begin{lstlisting}

\end{lstlisting}

\subsubsection{Exercice 3}
\textbf{Donnée : } Nous	pouvons	maintenant	analyser	les	données	collectées	par	perf	avec	la	commande	perf	report.	
Cette	commande	lance	un	outil	interactif	(interface	console de	type	“ncurses”):	il	est	donc	possible	
d'interagir avec	les	touches	du	clavier.Chaque	ligne	représente	une	fonction,	celles	récoltant	le	plus	d'évènements	sont	montrées	en	haut.	
En	vous	basant sur	les	informations	disponibles	dans	le	wiki	de	perf	
(\url{https://perf.wiki.kernel.org/index.php/Tutorial}),	décrivez	ce	que	représente	chaque	colonne.\\\\

\textbf{Exécution du code : } \\
\begin{lstlisting}

\end{lstlisting}

\subsubsection{Exercice 4}
\textbf{Donnée : } Sur	la	capture	ci-dessus,	on	voit	par	exemple	que	la	majorité	des	cycles	de	l'application	sont	passés	
dans	la	fonction	std::string::size() qui	est	contenue	dans	la	librairie	standard.	Il	nous	manque	
cependant	une	information	capitale :	quelle	fonction	de	notre	application	fait	appel	à	cette	fonction	?	
Grâce	à	l'option	-g passée	à	perf	record,	nous	pouvons	afficher	le	call-graph	complet.	Il	suffit	de	
naviguer	dans	l'interface	et	développer	la	ligne	std::string::size() avec	la	touche	“Enter”.	
Avec	les	instructions	précédentes,	déterminez quelle	fonction	de	notre	application	fait	(indirectement)	
appel	à	std::string::size().\\\\

\textbf{Exécution du code : } \\
\begin{lstlisting}

\end{lstlisting}

\subsubsection{Exercice 5}
\textbf{Donnée : } Maintenant	que	vous	savez	quelle	fonction	utilise	le	plus	de	ressources	CPU,	trouvez	une	optimisation	
du	code	permettant	de	réduire	drastiquement	le	temps	d'exécution	(vous	devriez	arriver	à	quelques	
dixièmes	de	secondes	pour	le	fichier	sample).\\\\
\textbf{Emplacement du code : } \textit{}\\

\textbf{Exécution du code : } \\
\begin{lstlisting}

\end{lstlisting}

