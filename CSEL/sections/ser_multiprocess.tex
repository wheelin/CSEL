\section{Programmation système : processus multiples}
\subsection{Exercice 1}
Concevez et développez une petite application mettant en oeuvre un des services de communication proposé par linux entre un processus parent et un processus enfant.
Le processus enfant devra émettre quelques messages sous forme de texte vers le processus parent, lequel les affichera sur la console. Le mesage exit permettra de terminer l'application.\\
Cette application devra impérativement capturer tous les signaux et les ignorer. Seul un message d'information sera affiché sur la console.\\
Chemin des sources : mulitprocess, ex1_mutliprocess\\
Voici l'output obtenu après lancement du programme:
\begin{lstlisting}
# ./app_a 
In the main process.

In the parent process

In the children process.

Enter a msg to send to the parent : bonojour
bonojour will be written to the parent.
Enter a msg to send to the parent : Received 50 from the children. String is : bonojour
salut
salut will be written to the parent.
Enter a msg to send to the parent : Received 50 from the children. String is : salut
hello
hello will be written to the parent.
Enter a msg to send to the parent : Received 50 from the children. String is : hello
exit
exit will be written to the parent.
Received 50 from the children. String is : exit
#
\end{lstlisting}
Le processus enfant demande à l'utilisateur de tapper quelque chose, qui est ensuite retransmis au parent, qui l'affiche. \\
Ensuite, un second output montre que le SIGINT est géré pour afficher un message en écrivant \textlt{Reveived signal is :} suivi du type de signal. Dans le cas suivant, c'est un ctrl-c qui a été envoyé. 
\begin{lstlisting}
Enter a msg to send to the parent : Received 50 from the children. String is : 
^CReceived signal is : Interrupt
Received signal is : Interrupt
 will be written to the parent.
Enter a msg to send to the parent : Received 50 from the children. String is : 
\end{lstlisting}

\subsection{Travail à réaliser}

