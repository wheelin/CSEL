\section{Travaux pratiques 1}
\subsection{Gestion de la mémoire, bibliothèques et fonctions utiles}
\subsubsection{Exercice 4}
\noindent
\textbf{Donnée : }Créer dynamiquement des éléments dans le noyau. Adapter un module noyau afin que l'on puisse lors de son installation spécifier un nombre d'éléments à créer ainsi qu'un texte initial à stocker dans les éléments précédemment alloués. Chaque élément contiendra également un numéro unique, Les éléments seront créés lors de l'installation du module et chainés dans une liste. Ces éléments seront détruits lors de la désinstallation du module. Des messages d'information seront émis afin de permettre le debugging du module.
\subsubsection{Exercice 5}
\noindent
\textbf{Donnée : }Indiquer les différents alocateurs SLAB disponibles dans le noyau Linux pour la cible ORDOID-XU3
\begin{enumerate}
	\item SLAB : "as cache frendly as possible, benchmark frendly"
	\item SLOB : "as compact as possible"
	\item SLUB : "Simple and instruction cost counts. Superior Debugging. Defragmentation. Execution time friendly"
\end{enumerate}
Source : \url{https://www.google.ch/url?sa=t&rct=j&q=&esrc=s&source=web&cd=3&cad=rja&uact=8&ved=0CDEQFjACahUKEwiqj6GfhKbIAhWLXBoKHXDUAow&url=http%3A%2F%2Fwww.cs.berkeley.edu%2F~kubitron%2Fcourses%2Fcs194-24-S14%2Fhand-outs%2Fbonwick_slab.pdf&usg=AFQjCNENx6yGi8dVRbdR2Si1OpXE_NuNkg&sig2=ZdJ_jUWHIfO1qFIIikEyHA}