\newpage
\section{Programmation noyau : Pilotes de périphériques}
\subsection{Pilotes orientés mémoire}
\subsubsection{Exercice 1}
\textbf{Donnée : } Réaliser	un	pilote	orienté	mémoire	permettant	de	mapper	en	espace	utilisateur	les	registres	de	la	
 FPGA	en	utilisant	le	fichier	virtuel	\textit{/dev/mem}.	Ce	pilote	permettra	de	lire	l’identification	du	uP	
(chip	id)	décrit	dans	l’exercice	no 6 du	cours	sur	la	programmation	de	modules	noyau.\\\\
\textbf{Complément : } Ce pilote n'est pas un module noyau, il faut prendre le Makefile d'un des codes d'exemples, par exemple Fibonacci. La cible doit tout de même être démarrée en mode nfs.\\\\
\textbf{Emplacement du code : } \textit{/PilotesPeripheriques/exercice1-mmap}\\

\textbf{Exécution du code : } \\
\begin{lstlisting}
# ./app_a                                                                       
uP register: Bit 31..12 : product id=0x65422                                    
uP register: Bit 11..8  : package id=0x0                                        
uP register: Bit 7..4   : major revision=0x0                                    
uP register: Bit 3..0   : minor revision=0x1 
\end{lstlisting}

\textbf{Remarques : } On obtient bien le même id qu'avec le module noyau de l'exercie 6 de la série précédente.\\
Pour utiliser correctement la commande mmap, il ne faut pas mettre l'offset à 0, mais à 0x1000000 (adresse du chipid de l'exercice6), sinon le code ne fonctionne pas, il essaie de lire une zone mémoire interdite ou inexistante.

\subsubsection{Exercice 2}
\textbf{Donnée : } Sur	la	base	de	l’exercice	1,	développer	un	pilote	orienté	caractère	permettant	de	mapper	en	
espace	utilisateur	ces registres	(implémentation	de	l’opération	de	fichier	« mmap »).	
Le	driver	orienté	mémoire	sera	ensuite	adapté	à	cette	nouvelle	interface.
Remarque :	à	effectuer	après	les	exercices	des	pilotes	orientés	caractère\\\\
\textbf{Emplacement du code : }\\ \textit{/PilotesPeripheriques/exercice2-mmapModule/user}\\
\textit{/PilotesPeripheriques/exercice2-mmapModule/noyau}\\

\textbf{Remarque : }Cet exercice a été compliqué à réaliser. Le code est un mélange de l'exercice 1 et 5. Il faut garder en tête que pour mapper les registres en espace utilisateur, il faut utiliser les fonction standard des file\_operations (open,close,mmap) et en plus, ajouter les vm\_operations\_struct (open,close) pour agir sur la zone mémoire. Dans le pilote orienté caractère, on utilise la fonction \textit{remap\_pfn\_range} pour mapper la zone mémoire.\\

\textbf{Exécution du code : } \\
\begin{lstlisting}
# modprobe mymodule                                                             
[  854.599130] [c2] mod: successfully loaded with major 249                     
# mknod /dev/mod c 249 0    
# ./app_a /dev/mod                                                              
[ 1166.619318] [c0] mod: open                                                   
[ 1166.620800] [c0] mod: mmap                                                   
[ 1166.623246] [c0] VMA open, virt b6f71000, phys 10000000                      
[ 1166.628986] [c0] VMA close.                                                  
[ 1166.631259] [c0] mod: release                                                
file /dev/mod open                                                              
Physical memory: Bit 31..12 : product id=0x65422                                
Physical memory: Bit 11..8  : package id=0x0                                    
Physical memory: Bit 7..4   : major revision=0x0                                
Physical memory: Bit 3..0   : minor revision=0x1                                

Virtual memory: Bit 31..12 : product id=0x43108                                 
Virtual memory: Bit 11..8  : package id=0x3                                     
Virtual memory: Bit 7..4   : major revision=0x2                                 
Virtual memory: Bit 3..0   : minor revision=0xa                                 
file /dev/mod close 
# modprobe -r mymodule                                                          
[ 1259.338388] [c0] mod: successfully unloaded   
\end{lstlisting}

\subsection{Pilotes orientés caractères}
\subsubsection{Exercice 3}
\textbf{Donnée : } Implémenter	un	pilote	de	périphérique	orienté	caractère.	Ce	pilote	sera	capable	de	stocker	dans	
une	variable	globale	au	module	les	données	reçues	par	l’opération	write et	de	les	restituer	par	
l’opération	read. Pour	tester	le	module,	on	utilisera	les commandes « echo » et	« cat ».\\\\
\textbf{Emplacement du code : } \textit{/PilotesPeripheriques/exercice3-ReadWrite}\\

\textbf{Exécution du code : } \\
\begin{lstlisting}
$ make clean all
$ sudo make install
\end{lstlisting}
\begin{lstlisting}
# modprobe mymodule                                                             
[ 4107.528630] [c2] mod: successfully loaded with major 249                     
# mknod /dev/mod c 249 0                                                        
# echo -n test module > /dev/mod                                                
[ 4134.907095] [c1] mod: open()                                                 
[ 4134.908540] [c1] mod: write test module                                      
[ 4134.912539] [c1] mod: release()                                              
# cat /dev/mod                                                                  
[ 4145.581863] [c0] mod: open()                                                 
[ 4145.583312] [c0] mod: read test module                                       
[ 4145.587140] [c0] mod: read test module                                       
[ 4145.590820] [c0] mod: release()                                              
# modprobe -r  mymodule                                                         
[ 4157.528779] [c1] mod: successfully unloaded   
\end{lstlisting}

\subsubsection{Exercice 4}
\textbf{Donnée : } Etendre	la	fonctionnalité	du	pilote	de	l’exercice	\#3 afin	que	l’on	puisse	à	l’aide	d’un	paramètre	
module	spécifier	le	nombre	d’instance.	Pour	chaque	instance	on	créera	une	variable	unique	
permettant	de	stocker	les	données	échangées	avec	l’application	en	espace	utilisateur.\\\\
\textbf{Emplacement du code : } \textit{/PilotesPeripheriques/exercice4-multi\_instance}\\

\textbf{Exécution du code : } \\
\begin{lstlisting}
# mknod /dev/mydev_0 c 248 0
# mknod /dev/mydev_1 c 248 1
# 
# echo -n "bonjour, monsieur" > /dev/mydev_0
[12232.424660] [c1] skeleton : open operation, major version : 248, minor version : 0
[12232.430827] [c1] skeleton : opened for reading and writing...
[12232.436572] [c1] minor : 0
[12232.436572] count = 17
[12232.441568] [c1] skeleton : efault value=-14
[12232.445802] [c1] skeleton : write operation... written=17
[12232.451198] [c1] skeleton : release operation
#
# echo -n "bien le bonjour, mademoiselle" > /dev/mydev_1
[12254.024382] [c1] skeleton : open operation, major version : 248, minor version : 1
[12254.030622] [c1] skeleton : opened for reading and writing...
[12254.036288] [c1] minor : 1
[12254.036288] count = 29
[12254.041274] [c1] skeleton : efault value=-14
[12254.045522] [c1] skeleton : write operation... written=29
[12254.050909] [c1] skeleton : release operation
# 
# cat /dev/mydev_0
[12278.473668] [c0] skeleton : open operation, major version : 248, minor version : 0
[12278.479815] [c0] skeleton : opened for reading and writing...
[12278.485576] [c0] skeleton : read operation... read=17
[12278.490663] [c6] skeleton : read operation... read=0
[12278.495527] [c6] skeleton : release operation
bonjour, monsieur
# 
# cat /dev/mydev_1
[12291.694063] [c0] skeleton : open operation, major version : 248, minor version : 1
[12291.700284] [c7] skeleton : opened for reading and writing...
[12291.706190] [c7] skeleton : read operation... read=29
[12291.710985] [c7] skeleton : read operation... read=0
[12291.715919] [c7] skeleton : release operation
bien le bonjour, mademoiselle
\end{lstlisting}

\subsubsection{Exercice 5}
\textbf{Donnée : } Développer	une	petite	application	en	espace	utilisateur	permettant	d’accéder	à	ces	pilotes	
orientés	caractère.	L’application	devra	écrire	un	texte	dans	le	pilote	et	le	relire.\\\\
\textbf{Complément : } Le module noyau a été repris de l'exercice 3.\\\\
\textbf{Emplacement du code : }\\ \textit{/PilotesPeripheriques/exercice5-UserAccess/user}\\
\textit{/PilotesPeripheriques/exercice5-UserAccess/noyau}\\\\
\textbf{Exécution du code : } \\
\begin{lstlisting}
lmi@csel1:~/workspace/csel1/environment/peripheral/exercice5/user$ make clean all
lmi@csel1:~/workspace/csel1/environment/peripheral/exercice5/user$ cd ../noyau/
lmi@csel1:~/workspace/csel1/environment/peripheral/exercice5/noyau$ make clean all
lmi@csel1:~/workspace/csel1/environment/peripheral/exercice5/noyau$ sudo make install
\end{lstlisting}
\begin{lstlisting}
# modprobe mymodule                                                             
[  231.111019] [c2] mod: successfully loaded with major 249    
# mknod /dev/mod c 249 0                  
# ./app_a /dev/mod HELLO                                                        
[  234.969067] [c0] mod: open()                                                 
[  234.970719] [c0] mod: write HELLO                                            
[  234.973927] [c0] mod: read HELLO                                             
[  234.977039] [c0] mod: release()                                              
file /dev/mod open                                                              
write HELLO                                                                     
read HELLO                                                                      
file /dev/mod close                                                             
# modprobe -r mymodule                                                          
[  261.512478] [c0] mod: successfully unloaded
\end{lstlisting}

\subsection{Opérations bloquantes}
\subsubsection{Exercice 6}
\textbf{Donnée : } Développer	un pilote	et	une	application	utilisant	les	entrées/sorties	bloquantes	pour	signaler	une	
interruption	matérielle provenant	de	l’un	des	switches	de	la	carte	d’extension	de	ODROID-XU3.
L’application	utilisera	le	service	select	pour	compter	le	nombre	d’interruptions.\\
\textit{Remarque :	les	switches	non	pas	d’anti-rebond,	par	conséquent il	est	fort	probable	que	vous	
comptiez	un	peu	trop	d’impulsions ;	effet	à	ignorer.	}
\\\\
\textbf{Remarque : }L'application utilisateur attend de recevoir 5 interruptions des boutons avant s'arrêter.\\\\
\textbf{Emplacement du code : }\\ \textit{/PilotesPeripheriques/exercice6-SelectButton/user}\\
\textit{/PilotesPeripheriques/exercice6-SelectButton/noyau}\\

\textbf{Exécution du code : } \\
\begin{lstlisting}
# modprobe mymodule                                                             
[   90.125005] [c2] mod: successfully loaded with major 249                     
[   90.129074] [c2] Interrupt handler module loaded in kernel                   
[   90.134167] Configuring pins[   90.137059] _gpio_request: gpio-176 (SW1) sta6
[   90.142048] [c2] _gpio_request: gpio-177 (SW2) status -16                    
[   90.147391] [c2] _gpio_request: gpio-168 (SW3) status -16                    
[   90.152778] [c2] _gpio_request: gpio-164 (SW4) status -16                    
[   90.158239] [c4] Configuring switches interrupts                             
[   90.162625] Pins and interrupts have been configured.[   90.167598] Init waie
# ./app_a /dev/mod                                                              
[   97.905744] [c0] mod: open                                                   
[   97.907156] [c0] Threads started                                             
file /dev/mod open                                                              
[  102.317965] [c0] Some switch has been pressed                                
[  102.320893] [c2] Thread awake                                                
Interrupt 1                                                                     
[  103.728868] [c0] Some switch has been pressed                                
[  103.731800] [c0] Thread awake                                                
Interrupt 2                                                                     
[  104.686291] [c0] Some switch has been pressed                                
Interrupt 3                                                                     
[  106.084775] [c0] Some switch has been pressed                                
[  106.087702] [c1] Thread awake                                                
Interrupt 4                                                                     
[  107.048387] [c0] Some switch has been pressed                                
[  107.051374] [c1] mod: release                                                
Interrupt 5                                                                     
[  107.707804] [c0] Some switch has been pressed                                
[  107.710718] [c0] Thread awake                                                
[  107.713669] [c1] Threads stopped                                             
file /dev/mod close                                                             
# modprobe -r mymodule                                                          
[  119.950452] [c0] Freeing interrupts                                          
[  119.952344] Interrupts freed[  119.955165] mod: successfully unloaded
\end{lstlisting}

\subsection{sysfs}
\subsubsection{Exercice 7}
\textbf{Donnée : } Développer	un	pilote	de	périphérique	orienté	caractère	permettant	de	valider	la	fonctionnalité	du	
sysfs.	Le	pilote		offrira	des	attributs	de	périphérique	afin	pouvoir	lire	et	écrire	un	bloc	de	données	
composé	de	quelques	membres	et	de	pouvoir	modifier	le	contenu	de	la	valeur	entière.	Seules	les	
commandes	« echo » et	« cat » doivent	être	nécessaire	pour	manipuler	ces	attributs.\\\\

\textbf{Emplacement du code : } \textit{/PilotesPeripheriques/exercice7-sysfs}\\


\textbf{Exécution du code : } \\
\begin{lstlisting}
# echo -n "salut" > /sys/devices/platform/skeleton/attr 
# 
# 
# cat /sys/devices/platform/skeleton/attr 
salut# 
\end{lstlisting}

\subsection{ioctl (optionnel)}
\subsubsection{Exercice 8}
Cet exercice n'a pas été réalisé.

\subsection{procfs (optionnel)}
\subsubsection{Exercice 9}
Cet exercice n'a pas été réalisé.

\subsection{Gestionnaires de périphériques}
\subsubsection{Exercice 10}
\textbf{Donnée : } Implémenter	à	l’intérieur	d’un	pilote	de	périphérique	orienté	caractère,	les	mécanismes	
nécessaires	à	la	création	du	fichier	d’accès	au	pilote	(remplacement	de	la	commande	« mknod »)	
par	l’utilitaire	« mdev »	de	la	BusyBox.\\\\
\textbf{Emplacement du code : } \textit{/PilotesPeripheriques/exercice10-periph\_handler}\\
Il est important de rappeler qu'il faut demander un numéro de device via le service \textit{alloc\_chrdev\_region} au noyau pour que ça marche correctement. Le module noyau préparé par nos soins a le numéro 249, il est bien présent dans le dossier \textit{/dev/}, mais aussi dans l'arborescence \textit{/sys/}. \textit{mdev} s'occupe de créer le nœud pour le driver quand celui-ci est chargé avec \textit{modprobe}.\\

\textbf{Exécution du code : } \\
\begin{lstlisting}
# cat /proc/devices 
Character devices:
  1 mem
  2 pty
  3 ttyp
  4 /dev/vc/0
  4 tty
  4 ttyS
  5 /dev/tty
  5 /dev/console
  5 /dev/ptmx
  7 vcs
 10 misc
 13 input
 21 sg
 29 fb
 81 video4linux
 89 i2c
116 alsa
128 ptm
136 pts
180 usb
188 ttyUSB
189 usb_device
204 ttySAC
212 DVB
226 drm
249 mySkeletonModule
250 bsg
251 iio
252 watchdog
253 media
254 rtc

Block devices:
259 blkext
  7 loop
  8 sd
 65 sd
 66 sd
 67 sd
 68 sd
 69 sd
 70 sd
 71 sd
128 sd
129 sd
130 sd
131 sd
132 sd
133 sd
134 sd
135 sd
179 mmc
254 device-mapper
# find /dev/ -iname "*ske*"
/dev/skeleton
# find /sys/ -iname "*ske*"
/sys/bus/platform/devices/skeleton
/sys/bus/platform/drivers/skeleton
/sys/bus/platform/drivers/skeleton/skeleton
/sys/devices/platform/skeleton
\end{lstlisting}








